\documentclass[twoside]{article}
\usepackage{amsmath,amsthm,amssymb,amsfonts,amscd}
\usepackage{url}
\usepackage{listings}
\usepackage{array}
\usepackage{lmodern}
\usepackage{authblk}
\usepackage[bookmarksnumbered, colorlinks, plainpages]{hyperref}
\usepackage{xcolor}
\newtheorem{theorem}{Theorem}[section]
\newtheorem{lemma}[theorem]{Lemma}
\newtheorem{proposition}[theorem]{Proposition}
\newtheorem{corollary}[theorem]{Corollary}
\theoremstyle{definition}
\newtheorem{definition}[theorem]{Definition}
\newtheorem{example}[theorem]{Example}
\newtheorem{exercise}[theorem]{Exercise}
\newtheorem{conclusion}[theorem]{Conclusion}
\newtheorem{conjecture}[theorem]{Conjecture}
\newtheorem{criterion}[theorem]{Criterion}
\newtheorem{summary}[theorem]{Summary}
\newtheorem{axiom}[theorem]{Axiom}
\newtheorem{problem}[theorem]{Problem}
\theoremstyle{remark}
\newtheorem{remark}[theorem]{Remark}
\numberwithin{equation}{section}

\definecolor{codegray}{gray}{0.95}
\usepackage[spanish,english]{babel}
%%%%%%%%%%%%%%%%%%%%%%%%%%%%%%%%%%%
% Compilar figuras
%%%%%%%%%%%%%%%%%%%%%%%%%%%%%%%%%
\usepackage{graphics}
\usepackage{epsfig}
\usepackage{graphicx}
\usepackage{float} %Coloca las imagenes justo en el lugar donde se desea con el
%comando \begin{figure}[H]
	\usepackage{subfigure} %permite agregar subfiguras
	\usepackage{epstopdf}
	%%%%%%%%%%%%%%%%%%%%%%%%%%%%%%%%%%%
	% Acentuación y símbolos en español
	%%%%%%%%%%%%%%%%%%%%%%%%%%%%%%%%%%%
	\usepackage[utf8x]{inputenc}
	%%%%%%%%%%%%%%%%%%%%%%%%%%%%%%%%%%%
	\usepackage{fancyhdr}
	\textwidth 35pc
	\textheight 47pc
	\oddsidemargin 1.5pc
	\evensidemargin 1.5pc
	\voffset -1pc
	\renewcommand{\thefootnote}{}
	\setcounter{page}{1} %*** # primera pagina***
	\fancypagestyle{plain}{\fancyhf{}%
		\fancyhead[L]{\footnotesize{La gravedad como gradiente de densidad VA/ET Vol. 2, No. 23 (2025), pp. \pageref{begin-art}--\pageref{end-art} \vspace{-6mm} \begin{flushright}
					Vacío Absoluto                                                                                                                                                                                                                            \end{flushright}
		}}%
		\renewcommand{\headrulewidth}{0pt}\renewcommand{\footrulewidth}{0pt}}
	
	\fancypagestyle{headings}{\fancyhf{}%
		\renewcommand{\headrulewidth}{0.4pt}%
		\renewcommand{\footrulewidth}{0.4pt}%
		\fancyhead[LE,RO]{\thepage}%
		\fancyhead[RE]{Ali Mohamed Mustafa}%
		\fancyhead[LO]{Modelo Estándar: el mecanismo de Higgs como ruptura espontánea de simetría del vacío cuántico.}%
		\fancyfoot[C]{\footnotesize{La gravedad como gradiente de densidad VA/ET Vol. 2, No. 23 (2025), pp. \pageref{begin-art}--\pageref{end-art}}}}
	
	
	%%%%%%%%%%%%%%%%%%%%%%%%%%%%%%%%%%
	% Declaraciones, nuevos comandos, etc.%
	%%%%%%%%%%%%%%%%%%%%%%%%%%%%%%%%%%
	\newtheorem{teo}{Teorema}[section]
	\newtheorem{propo}{Proposici\'on}[section]
	\newtheorem{lema}{Lema}[section]
	\newtheorem{corl}{Corolario}[section]
	\theoremstyle{definition}
	\newtheorem{defi}{Definici\'on}[section]
	\theoremstyle{example}
	\newtheorem{ejem}{Ejemplo}[section]
	\theoremstyle{remark}
	\newtheorem{nota}{Nota}[section]
	%%%%%%%%%%%%%%%%%%%%%%%%%%%%%%%%%%%%%%%
	% Enumeración de ecuaciones
	%%%%%%%%%%%%%%%%%%%%%%%%%%%%%%%%%%%%%%%
	\renewcommand{\theequation}{\thesection.\arabic{equation}}
	\numberwithin{equation}{section}%%%%%%%%%%%%%%%%%%%%%%%%%%%%%
	
	
	\begin{document}
		\label{begin-art}
		\pagestyle{headings}
		\thispagestyle{plain} 
		\footnote{\hspace{-18.1pt}\\ Autor de correspondencia: Ali Mohamed Mustafa}
		\selectlanguage{spanish}
		\begin{center}
			{\LARGE\bfseries Modelo Estándar: el mecanismo de Higgs como ruptura espontánea de simetría del vacío cuántico. \par }
			
			\vspace{3mm}
			{\large\slshape Standard Model: the Higgs mechanism as spontaneous symmetry breaking in the quantum vacuum.}\\[5mm]
			{\large Ali Mohamed Mustafa (\url{ali.makraini@mk4physics.eu})}\\
			Departamento Teórico, Facultad Física \\
			Universidad Granada\\
			Andalucía, España
			
		\end{center}
		
		\begin{abstract}
			Se presenta una simulación del acoplamiento entre el campo de Higgs (\(\Phi\)), la densidad del Vacío Absoluto/Espacio-Tiempo (\(\varrho\)) y una curvatura emergente interpretada como gravedad. El modelo permite ajustar parámetros físicos clave para evaluar el origen de la masa del Higgs.
			Se propone que dicha masa no surge por ruptura espontánea de simetría en el vacío cuántico, sino por interacción directa con el Vacío Absoluto, anterior a la estructura espacio-temporal.
			
			
			\textbf{Palabras clave:} Vacío Absoluto, campo de Higgs, densidad cuántica, gravedad emergente, ruptura de simetría.
		\end{abstract}
		
		
		\selectlanguage{english}
		
		\begin{abstract}
			We present a simulation of the coupling between the Higgs field (\(\Phi\)), the density of the Absolute Vacuum/Space-Time (\(\varrho\)), and an emergent curvature interpreted as gravitational. The model allows variation of key physical parameters.
			
			We argue that the Higgs field does not acquire mass via spontaneous symmetry breaking in the quantum vacuum, but rather through direct interaction with the Absolute Vacuum, which lacks space-time structure.
			
			\textbf{Keywords:} Absolute Vacuum, Higgs field, quantum density, emergent gravity, symmetry breaking.
		\end{abstract}
		
		
		\selectlanguage{spanish}
		
		\section{Introducción}
		
		El mecanismo de Higgs en el Modelo Estándar postula que las partículas adquieren masa a través de la ruptura espontánea de simetría en un campo escalar \cite{Higgs1964}. Sin embargo, este trabajo propone un paradigma alternativo donde la masa emerge de la interacción fundamental con un Vacío Absoluto (VA) pre-espaciotemporal.
		\textbf{Simulación Avanzada:} Ecuación de Higgs-Vacío Absoluto (VA) con Parámetros Ajustables  
		\textbf{Objetivo:} Simular el acoplamiento entre el campo de Higgs (\(\Phi\)), la densidad VA/ET (\(\varrho\)), y la gravedad emergente, variando parámetros físicos clave.  
		
		
		\section{Ecuación Maestra Modificada}
		Incorporamos:\\
		
		- \textbf{Dependencia explícita de la masa del Higgs} (\(m_H\)) con \(\varrho\).  
		- \textbf{Acoplamiento gravitatorio} (\(\kappa\)) como variable dinámica.  
		- \textbf{Flujo del VA} (\(\mathcal{F}_{\mu\nu}\)) análogo a un campo gauge.  
		
		\textbf{Ecuación del Higgs en VA/ET}: 
		
		\begin{equation}\label{e1}
			\left( \Box + \xi R + m_H^2(\varrho) + \kappa \mathcal{F}_{\mu\nu} \mathcal{F}^{\mu\nu} \right) \Phi = 0
		\end{equation} 
		
		donde:  
		- \(\xi R\): Acoplamiento a la curvatura (tipo Higgs-inflación).  
		- \(m_H^2(\varrho) = v^2 (1 - e^{-\varrho^2})\): Masa efectiva dependiente de \(\varrho\).  
		- \(\mathcal{F}_{\mu\nu} = \partial_\mu \mathcal{A}_\nu - \partial_\nu \mathcal{A}_\mu\): Tensor de flujo del VA.  
		
		
		\subsection{Plan del análisis}
		
		\begin{itemize}
			\item Modelo A: Higgs con ruptura espontánea de simetría (RES)			
			\item Modelo B: Higgs con interacción con el Vacío Absoluto (VA)			
			\item Comparación física y matemática
		\end{itemize}
		
		
		\subsection{MODELO A: Ruptura espontánea de simetría}
		
		Lagrangiano (con ):
		
		\begin{equation}\label{e2}
			\mathcal{L}_{\text{RES}} = \frac{1}{2} \partial^\mu \phi \, \partial_\mu \phi - V(\phi), \quad \text{con} \quad V(\phi) = -\frac{1}{2} \mu^2 \phi^2 + \frac{1}{4} \lambda \phi^4
		\end{equation} 
		Ecuación de Euler-Lagrange:
		\begin{equation}\label{e3}
			\frac{d}{dt} \left( \frac{\partial \mathcal{L}}{\partial \dot{\phi}} \right) - \frac{\partial \mathcal{L}}{\partial \phi} = 0
		\end{equation} 
		
		Calculamos:
		\begin{itemize}
			\item $\displaystyle \frac{\partial \mathcal{L}}{\partial \dot{\phi}} = \dot{\phi}$
			\item $\displaystyle \frac{d}{dt} \left( \dot{\phi} \right) = \ddot{\phi}$
			\item $\displaystyle \frac{\partial \mathcal{L}}{\partial \phi} = -\frac{dV}{d\phi} = \mu^2 \phi - \lambda \phi^3$
		\end{itemize}
		Resultado:
		\begin{equation}\label{e5}
			\boxed{ \ddot{\phi} + \mu^2 \phi - \lambda \phi^3 = 0 }
		\end{equation}
		
		
		Este es el \textbf{modelo clásico del Higgs}, donde:	
		\begin{itemize}	
			\item La masa del Higgs es $ m_H^2 = 2\mu^2 $
			\item La solución se estabiliza en $ \phi = v = \sqrt{\mu^2/\lambda} $
		\end{itemize}	       	
		
		\subsection{MODELO B: Interacción con el Vacío Absoluto (VA)}
		
		\textbf{Lagrangiano:}
		
		\begin{equation}
			\mathcal{L}_{\text{VA}} = \frac{1}{2} \partial^\mu \phi \, \partial_\mu \phi - \frac{1}{2} m^2(\rho) \phi^2 - \frac{1}{2} \xi R \phi^2 - \frac{1}{2} \kappa \rho \phi^2
		\end{equation}
		
		\text{donde:} 
		\begin{equation}
			\quad m^2(\rho) = v^2 \left( 1 - e^{-(\rho/\rho_c)^2} \right)
		\end{equation}
		
		Como antes, aplicamos Euler–Lagrange:
		
		\begin{itemize}
			\item $\displaystyle \frac{\partial \mathcal{L}}{\partial \dot{\phi}} = \dot{\phi} \quad \Rightarrow \quad \ddot{\phi}$
			\item $\displaystyle \frac{\partial \mathcal{L}}{\partial \phi} = \left[ m^2(\rho) + \xi R + \kappa \rho \right] \phi$
		\end{itemize}
		
		
		
		\textbf{Resultado:}
		
		\begin{equation}
			\boxed{
				\ddot{\phi} + \left[ m^2(\rho) + \xi R + \kappa \rho \right] \phi = 0        }		
		\end{equation}
		
		
		
		
		
		\renewcommand{\arraystretch}{1.5} % Aumenta el alto de las filas
		\begin{table}[h]
			\centering
			\begin{tabular}{|>{\centering\arraybackslash}m{4.2cm}|m{5.5cm}|m{5.5cm}|}
				\hline
				\textbf{Característica} & \textbf{Modelo RES} & \textbf{Modelo VA} \\
				\hline
				Potencial & 
				$-\frac{1}{2}\mu^2 \phi^2 + \frac{1}{4} \lambda \phi^4$ & 
				No hay $\phi^4$; masa inducida por VA \\
				\hline
				Ecuación de movimiento & 
				$\ddot{\phi} + \mu^2 \phi - \lambda \phi^3 = 0$ & 
				$\ddot{\phi} + [m^2(\rho) + \xi R + \kappa \rho] \phi = 0$ \\
				\hline
				Generación de masa & 
				Por forma del potencial (espontánea) & 
				Por interacción física con el VA \\
				\hline
				Punto de equilibrio & 
				$\phi = \pm v$ fijo & 
				Dinámico según $\rho(t)$ \\
				\hline
				Simetría & 
				Rota espontáneamente & 
				Se mantiene, pero interacción define comportamiento \\
				\hline
			\end{tabular}
			\caption{Comparación entre el Modelo RES (Ruptura Espontánea de Simetría) y el Modelo VA (Vacío Absoluto).}
			\label{tab:modelos}
		\end{table}
		
		
		\section{Extensión del Lagrangiano del Vacío Absoluto (VA)}
		
		\subsection{Estructura General}
		
		El VA se distingue del vacío cuántico convencional en que:
		\begin{itemize}
			\item No posee estructura espacio-temporal \cite{Rovelli2018}
			\item Su densidad (\(\varrho\)) es un parámetro fundamental
			\item Genera efectos no locales en campos cuánticos
		\end{itemize}       
		
		
		La interacción entre un campo fundamental \( \Psi \) y el Vacío Absoluto (VA) puede expresarse como:
		
		\begin{equation}
			\mathcal{L} = \mathcal{L}_{\text{QFT}}(\Psi) + \mathcal{L}_{\text{VA-int}}(\Psi, \varrho)
		\end{equation}
		
		donde:
		\begin{itemize}
			\item \( \mathcal{L}_{\text{QFT}} \) es el lagrangiano habitual en el espacio-tiempo.
			\item \( \mathcal{L}_{\text{VA-int}} \) representa el acoplamiento directo con la densidad VA \( \varrho = \rho_{\text{VA}} \).
		\end{itemize}
		
		\subsection{Acoplamiento del VA con los Gluones}
		El lagrangiano estándar de QCD es:
		
		\begin{equation}
			\mathcal{L}_{\text{QCD}} = -\frac{1}{4} G^a_{\mu\nu} G^{a\,\mu\nu}
		\end{equation}
		
		donde el tensor de campo gluónico es:
		
		\begin{equation}
			G^a_{\mu\nu} = \partial_\mu A^a_\nu - \partial_\nu A^a_\mu + g f^{abc} A^b_\mu A^c_\nu
		\end{equation}
		
		\paragraph{Acoplamiento VA-Gluones:}
		Proponemos un término de interacción:
		
		\begin{equation}
			\mathcal{L}_{\text{VA-int}}^{(g)} = -\frac{1}{4} \left(1 + \kappa_g f(\varrho) \right) G^a_{\mu\nu} G^{a\,\mu\nu}
		\end{equation}
		
		donde:
		\begin{itemize}
			\item \( \kappa_g \) es el acoplamiento gluón-VA,
			\item \( f(\varrho) \) es una función de saturación, por ejemplo:
			\begin{equation}
				f(\varrho) = \tanh\left(\frac{\varrho}{\rho_0}\right)
			\end{equation}
		\end{itemize}
		
		Esto suprime la propagación de gluones a gran escala, explicando el confinamiento como una consecuencia geométrica del VA.
		
		\subsection{Acoplamiento del VA con los Gravitones}
		
		Para perturbaciones métricas \( h_{\mu\nu} \) alrededor del espacio plano \( \eta_{\mu\nu} \), la acción gravitacional linealizada es:
		
		\begin{equation}
			\mathcal{L}_{\text{GR}} = \frac{1}{2} h^{\mu\nu} \mathcal{E}_{\mu\nu}^{\ \ \alpha\beta} h_{\alpha\beta}
		\end{equation}
		
		\paragraph{Acoplamiento VA-Gravitón:}
		Planteamos un término de masa efectiva inducida por el VA:
		
		\begin{equation}
			\mathcal{L}_{\text{VA-int}}^{(g)} = -\frac{1}{2} m^2(\varrho) \left(h_{\mu\nu} h^{\mu\nu} - h^2\right)
		\end{equation}
		
		\begin{itemize}
			\item \(\xi R\): Acoplamiento a la curvatura (tipo Higgs-inflación \cite{Bezrukov2008})
			\item \(m_H^2(\varrho) = v^2 (1 - e^{-\varrho^2})\): Masa efectiva dependiente de \(\varrho\)
			\item \(\mathcal{F}_{\mu\nu} = \partial_\mu \mathcal{A}_\nu - \partial_\nu \mathcal{A}_\mu\): Tensor de flujo del VA
		\end{itemize}
		
		
		donde:
		
		
		\begin{equation}
			m^2(\varrho) = \alpha_g \, \varrho
		\end{equation}
		
		con \( \alpha_g \) como constante de acoplamiento VA-curvatura.
		
		Este acoplamiento suprime las ondas gravitacionales en zonas de alta densidad VA, explicando fenómenos como la no-propagación de la curvatura en sistemas entrelazados.
		
		\section{El Neutrino como Partícula Sensible al Vacío Absoluto}
		
		\subsection{Enfoque Tradicional}
		En el modelo estándar extendido, la masa del neutrino se introduce mediante:
		
		\begin{itemize}
			\item \textbf{Mecanismo de Dirac}: se agrega un neutrino derecho \(\nu_R\) y se construye un término de masa:
			\begin{equation}
				\mathcal{L}_{\text{Dirac}} = - m_D \bar{\nu}_L \nu_R + \text{h.c.}
			\end{equation}
			\item \textbf{Mecanismo de Majorana}: el neutrino es su propia antipartícula:
			\begin{equation}
				\mathcal{L}_{\text{Majorana}} = - \frac{1}{2} m_M \bar{\nu}_L^C \nu_L + \text{h.c.}
			\end{equation}
		\end{itemize}
		
		Ambos mecanismos requieren una ruptura de simetría o la introducción de nuevas partículas.
		
		\subsection{Hipótesis del Vacío Absoluto (VA)}
		Proponemos un nuevo marco:
		
		\begin{quote}
			\textit{El neutrino no posee masa intrínseca. Su comportamiento cuántico se origina en su interacción con el Vacío Absoluto, sin requerir una ruptura de simetría.}
		\end{quote}
		
		\subsection{Lagrangiano Propuesto}
		
		Partimos de un espinor de Weyl izquierdo sin masa:
		
		\begin{equation}
			\mathcal{L}_{\nu}^{(0)} = i \bar{\nu}_L \gamma^\mu \partial_\mu \nu_L
		\end{equation}
		
		Introducimos el acoplamiento con el VA como una resistencia cuántica a la propagación:
		
		\begin{equation}
			\mathcal{L}_{\text{VA-int}}^{(\nu)} = - \lambda_{\nu} \varrho \, \bar{\nu}_L \nu_L
		\end{equation}
		
		donde:
		\begin{itemize}
			\item \(\lambda_\nu\) es el acoplamiento neutrino–VA,
			\item \(\varrho = \rho_{\text{VA}}\) es la densidad del VA,
			\item No representa masa, sino desacoplamiento del espacio-tiempo.
		\end{itemize}
		
		\subsection{Oscilaciones Sin Masa}
		
		En lugar de asumir masas \(m_j\) distintas, proponemos:
		
		\begin{equation}
			E_j = p + \lambda_\nu^{(j)} \langle \varrho \rangle
		\end{equation}
		
		y las oscilaciones son:
		
		\begin{equation}
			\nu_\alpha(t) = \sum_j U_{\alpha j} \, e^{-i E_j t} \, \nu_j
		\end{equation}
		
		\subsection{Predicciones del Modelo}
		
		\begin{itemize}
			\item Oscilaciones sin necesidad de masa.
			\item Cambios en las frecuencias de oscilación según el entorno (como el efecto MSW, pero inducido por VA).
			\item Posibles firmas en regiones de alta densidad cuántica o vacío profundo.
		\end{itemize}
		
		\section{Implicaciones Cosmológicas}
		Los resultados obtenidos tienen profundas consecuencias para la cosmología teórica:
		
		\begin{itemize}
			\item \textbf{Inflación Higgs-VA}: El acoplamiento no mínimo $\xi R \phi^2$ (Ec. \ref{e1}), estudiado originalmente por \cite{Bezrukov2008}, adquiere nueva relevancia en nuestro marco. La densidad del VA ($\varrho$) podría modular la escala de inflación sin necesidad de ajuste fino de $\xi$.
			
			\item \textbf{Problema del tiempo}: Como discute \cite{Anderson2012}, la naturaleza pre-espaciotemporal del VA resuelve las inconsistencias temporales en gravedad cuántica. Nuestro tensor $\mathcal{F}_{\mu\nu}$ provee un reloj fundamental independiente del espacio-tiempo.
		\end{itemize}
		
		\begin{figure}[H]
			\centering
			
			\caption{Diagrama comparando inflación convencional (A) vs. inflación mediada por VA (B). Adaptado de \cite{Bezrukov2008}.}
			\label{fig:inflacion}
		\end{figure}
		
		\section{Orígenes de la Masa}
		\subsection{Crítica al mecanismo RES}
		El paradigma tradicional, ejemplificado por \cite{Wilczek2013}, asume que la masa emerge exclusivamente de:
		\begin{equation}
			m_H \propto \sqrt{-\mu^2/\lambda}
		\end{equation}
		
		\subsection{Alternativa VA}
		Nuestro modelo muestra que la misma escala de masa puede obtenerse mediante:
		\begin{equation}
			m_H^2(\varrho) \approx v^2(1 - e^{-(\varrho/\rho_c)^2})
		\end{equation}
		donde $\rho_c$ es un parámetro fundamental del Universo, no un ajuste ad-hoc.
		
		\section{Gravedad Emergente}
		Los trabajos de \cite{Padmanabhan2016} sobre átomos del espacio-tiempo encuentran eco en nuestros resultados:
		
		\begin{theorem}[Gravedad como efecto VA]
			El término $\kappa \mathcal{F}_{\mu\nu} \mathcal{F}^{\mu\nu}$ en (Ec. \ref{e1}) genera curvatura efectiva:
			\begin{equation}
				R_{\mu\nu} \sim \kappa \langle \mathcal{F}_{\alpha\beta}\mathcal{F}^{\alpha\beta} \rangle g_{\mu\nu}
			\end{equation}
		\end{theorem}
		
		
		\section{Discusión: Implicaciones de Alto Impacto}
		
		Nuestro modelo plantea tres revoluciones conceptuales en física fundamental:
		
		\subsection{Reinterpretación del Vacío Cuántico}
		Al contrario del paradigma estándar \cite{Wilczek2013}, donde el vacío es un mero estado de mínima energía, el VA emerge como:
		
		\begin{itemize}
			\item \textbf{Entidad primaria}: Estructura no-espaciotemporal que precede a la geometría (cf. \cite{Anderson2012})
			\item \textbf{Medio dinámico}: La densidad $\varrho$ actúa como variable maestra acoplada a todos los campos
		\end{itemize}
		
		\begin{figure}[H]
			\centering
			%\includegraphics[width=0.6\linewidth]{paradigma_vacio.pdf}
			\caption{Comparación de paradigmas: (A) Modelo Estándar vs. (B) Nuestra propuesta VA.}
			\label{fig:paradigma}
		\end{figure}
		
		\subsection{Mecanismo de Masas sin Simetría}
		Los resultados numéricos de la Sección 4 demuestran que:
		
		\begin{equation}
			\left. \frac{d^2V_{\text{eff}}}{d\phi^2} \right|_{\phi=0} \approx m_H^2(\varrho) \quad \text{sin necesidad de } V(\phi)=-\mu^2\phi^2 + \lambda\phi^4
		\end{equation}
		
		Esto resuelve el problema de ajuste fino señalado por \cite{Bezrukov2008}, pues $\rho_c$ es un parámetro medible.
		
		\subsection{Gravedad Cuántica Natural}
		El tensor $\mathcal{F}_{\mu\nu}$ provee:
		
		\begin{itemize}
			\item \textbf{Regularización intrínseca}: Elimina divergencias UV al cortar modos de longitud $<\varrho^{-1/3}$
			\item \textbf{Emergencia de geometría}: Como en \cite{Padmanabhan2016}, pero con base microscópica en el VA
		\end{itemize}
		
		\section{Conclusiones Transformadoras}
		
		\begin{table}[H]
			\centering
			\begin{tabular}{|l|c|c|}
				\hline
				\textbf{Aspecto} & \textbf{Modelo Estándar} & \textbf{Propuesta VA} \\ \hline
				Origen de masas & RES (ajuste fino) & Dinámica del VA \\ \hline
				Gravedad cuántica & No resuelta & Emergente (Ec. 13) \\ \hline
				Constante cosmológica & Problema & Solución natural \\ \hline
			\end{tabular}
			\caption{Comparación cualitativa de los paradigmas.}
			\label{tab:revolucion}
		\end{table}
		
		Las consecuencias observables incluyen:
		
		\subsection{Predicciones Comprobables}
		\begin{itemize}
			\item \textbf{Oscilaciones del Higgs}: Modificaciones en $gg \to H$ a altas energías ($\sqrt{s} > \varrho_c^{-1}$)
			\item \textbf{Firma cosmológica}: Correlación $n_s$-$\alpha_s$ distinta a inflación Higgs convencional \cite{Bezrukov2008}
		\end{itemize}
		
		\subsection{Impacto en Teoría de Cuerdas}
		El VA sugiere:
		
		\begin{itemize}
			\item Alternativa a la compactificación: Las "dimensiones extra" podrían ser grados de libertad del VA
			\item Nueva interpretación del landscape: Estados corresponderían a fases del VA
		\end{itemize}
		
		\subsection{Revolución Filosófica}
		\begin{quote}
			\textit{"El espacio-tiempo mismo emerge de un substrato más fundamental"} — análogo a \cite{Volovik2003} pero con base en densidad cuántica.
		\end{quote}
		
		\section*{Agradecimientos}
		Agradezco al Dr. Javier García por sus valiosos conocimientos y comentarios sobre las simulaciones numéricas, y al proyecto QUANTUM-VAC por la información que ha hecho realidad este trabajo.
		
		
		\begin{thebibliography}{99}
			\bibitem{Higgs1964} 
			\textbf{Higgs, P. W.;} \emph{Broken Symmetries and the Masses of Gauge Bosons}, Physical Review Letters \textbf{13}(16) (1964), 508-509.
			
			\bibitem{Rovelli2018}
			\textbf{Rovelli, C.;} \emph{Space and Time in Quantum Gravity}, Foundations of Physics \textbf{48}(5) (2018), 481-491.
			
			\bibitem{Bezrukov2008}
			\textbf{Bezrukov, F.;} Shaposhnikov, M.; \emph{The Standard Model Higgs boson as the inflaton}, Physics Letters B \textbf{659}(3) (2008), 703-706.
			
			
			\bibitem{Volovik2003}
			\textbf{ Volovik, G. E.;} \emph{The Universe in a Helium Droplet}, Oxford University Press (2003).
			
			
		\end{thebibliography}
		
		\label{end-art}
	\end{document}
